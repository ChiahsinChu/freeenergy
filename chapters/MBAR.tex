\subsection{Multistate Bennett Acceptance Ratio\label{Sec:FEM:MBAR}}
\begin{chapquote}{Ernest Rutherford%, \textit{\url{https://en.wikiquote.org/wiki/Ernest_Rutherford}}
	}
	``An alleged scientific discovery has no merit unless it can be explained to a barmaid.''
\end{chapquote}
\begin{chapquote}{Yihan Shao
	}
	``So, you can never be a good scientist unless you go to bar regularly.''
\end{chapquote}
The Multistate Bennett Acceptance Ratio (MBAR) method was developed by Shirts and Chodera in 2008.\cite{ShirtsJCP2008} The following derivation quite follows Ref.~\cite{ShirtsarXiv2017}
Imaging you have carried out a series of simulations such as umbrella sampling, or replica exchange molecular dynamics simulations.
Now you have $K$ trajectories in total and each trajectory is characterized by Hamiltonian $H_k$ and inverse temperature $\beta_k$. The trajectories unnecessarily have the same number of conformations. Instead, the number of conformations in trajectory $k$ is $N_k$. Now, you mix all the samples and randomly pick one sample out of them. The probably for this sample to have coordinates $\mathbf{R}$ is 
\begin{equation}
p_m(\mathbf{R})=\frac 1N\sum_{k=1}^{K}N_kp_k(\mathbf{R}),
\end{equation}
in which $N=\sum\limits_{k=1}^{K}N_k$ and the subscript $m$ means mixed ensemble. $p_k(\mathbf{R}_n)$ is the probability of finding this snapshot in trajectory $k$, which satisfies
\begin{equation}
p_k(\mathbf{R})={c_k}^{-1}q_k(\mathbf{R}).
\end{equation}
$c_k$ is the normalization constant. You can see that this mixed ensemble does not follow Boltzmann statistics, even if $q_k$ does. It can be proved that if $p_k$ is normalized, then $p_m$ is also normalized.

The expectation of any operator $\hat{O}$ averaged over this mixed ensemble can be calculated by
\begin{equation}
\langle O\rangle_m=\int O(\mathbf{R})p_m(\mathbf{R})d\mathbf{R}\approx\frac 1N \sum\limits_{n=1}^NO(\mathbf{R}_n).
\end{equation}
Using energy reweighting\cite{TorrieJComputP1977}, we can calculate the expectation of this operator under any other Hamiltonian $H_i$ and probability $p_i$, which can be expressed as
\begin{align}
\langle O\rangle_i=&\int O(\mathbf{R})p_i(\mathbf{R})d\mathbf{R}\notag\\
                  =&\int O(\mathbf{R})\frac{p_i(\mathbf{R})}{p_m(\mathbf{R})}p_m(\mathbf{R})d\mathbf{R}\notag\\
                  \approx&\frac 1N\sum\limits_{n=1}^N O(\mathbf{R}_n)\frac{p_i(\mathbf{R}_n)}{p_m(\mathbf{R}_n)}\notag\\
                  =&\frac 1N\sum\limits_{n=1}^N O(\mathbf{R}_n){c_i}^{-1}\frac{q_i(\mathbf{R}_n)}{p_m(\mathbf{R}_n)}\notag\\
                  =&\sum\limits_{n=1}^N O(\mathbf{R}_n){c_i}^{-1}\frac{q_i(\mathbf{R}_n)}{\sum_{k=1}^{K}N_kp_k(\mathbf{R}_n)}
                  \label{Eq:FEM:MBAR:Oexpect}
\end{align}
Let $O=1$, we find
\begin{equation}
1=\sum\limits_{n=1}^N {c_i}^{-1}\frac{q_i(\mathbf{R}_n)}{\sum_{k=1}^{K}N_kp_k(\mathbf{R}_n)}.
\end{equation}
Since $c_i$ does not depend on $n$,
\begin{align}
c_i=&\sum\limits_{n=1}^N \frac{q_i(\mathbf{R}_n)}{\sum_{k=1}^{K}N_kp_k(\mathbf{R}_n)}\notag\\
   =&\sum\limits_{n=1}^N \frac{q_i(\mathbf{R}_n)}{\sum_{k=1}^{K}N_k{c_k}^{-1}q_k(\mathbf{R}_n)}
   \label{Eq:FEM:MBAR:c_i}
\end{align}
In Boltzmann statistics, $q_k(\mathbf{R})=\exp{\left[-\beta_kU_k(\mathbf{R})\right]}$ and $c_k=\int q_k(\mathbf{R})d\mathbf{R}$ is the partition function or the normalization constant. \textit{Note that we have not assumed anything about the statistics of ensemble $k$ and $i$. Besides, $i$ is unnecessarily within \{$k$\}. For instance, if we run replica exchange molecular dynamics simulations at $K$ inverse temperatures $\beta_1,\dots,\beta_K$, $\beta_i$ can be either one of these inverse temperatures or any other inverse temperature between $\beta_1$ and $\beta_K$. But extrapolation to inverse temperatures outside the range of $\left[\beta_K,\beta_1\right]$ is not recommended.}

If $q_k$ and $q_i$ follow Boltzmann statistics, and we define free energy $f_i=-{\beta_i}^{-1}\ln{c_i}$, Eq.~\ref{Eq:FEM:MBAR:c_i} becomes
\begin{equation}
f_i=-{\beta_i}^{-1}\ln{\sum\limits_{n=1}^N \frac{\exp{\left(-\beta_i U_i(\mathbf{R}_n)\right)}}{\sum_{k=1}^{K}N_k\exp{\left(\beta_kf_k-\beta_kU_k(\mathbf{R}_n)\right)}}},
\label{Eq:FEM:MBAR:f_i_final}
\end{equation}
which must be solved self-consistently and can be determined up to a constant. We can fix $f_1$ (to 0 usually). 

Again, from Eq.~\ref{Eq:FEM:MBAR:Oexpect}, we can define
\begin{equation}
W_{in}={c_i}^{-1}\frac{q_i(\mathbf{R}_n)}{\sum_{k=1}^{K}N_k{c_k}^{-1}q_k(\mathbf{R}_n)},
\end{equation}
which is the weight of snapshot $n$ in ensemble $i$ determined by Hamiltonian $H_i$ and temperature $\beta_i$. Specifically, for the Boltzmann statistics,
\begin{align}
W_{in}=\frac{e^{-\beta_i \left[U_i(\mathbf{R}_n)-f_i\right]}}{\sum_{k=1}^{K}N_k e^{-\beta_k \left[U_k(\mathbf{R}_n)-f_k\right]}}.
\end{align}

