\subsection{Thermodynamic Integration\label{Sec:FEM:TI}}
Thermodynamic Integration (TI) method was originally put forward by Kirkwood.\cite{KirkwoodJCP1935}. 
	
If the free energy, $A$, is a continuous function of $\lambda$ then we can write:
\begin{equation}
\Delta A = \int_{0}^{1} \frac{\partial{A(\lambda)}}{\partial{\lambda}} d\lambda
\label{Eq:deltaA1TI}
\end{equation} 
Now,
\begin{equation}
A(\lambda) = -k_{B}T\ln Q(\lambda)
\label{Eq:Alambda}
\end{equation} 
Thus,
\begin{align}
\Delta A =& -k_{B}T\int_{0}^{1} \left[ \frac{\partial{lnQ(\lambda)}}{\partial{\lambda}} \right]d\lambda \\
=&\int_{0}^{1}\frac{-k_{B}T}{Q(\lambda)}\frac{\partial{Q(\lambda)}}{\partial{\lambda}}d\lambda
\label{Eq:deltaA2TI}
\end{align} 
From the definition of Q:
\begin{equation}
Q_{NVT} = \frac{1}{{h}^{3N}N!} \iint \exp[-\beta H(\textbf{x},\textbf{p}_{x})] d\textbf{x}d\textbf{p}_\textbf{x},
\label{Eq:PFTI}
\end{equation}
we can write the following for $\frac{\partial{Q(\lambda)}}{\partial{\lambda}}$:
\begin{equation}
\frac{\partial{Q(\lambda)}}{\partial{\lambda}} = \frac{1}{{h}^{3N}N!} \iint \frac{\partial}{\partial{\lambda}}\exp[-\beta H(\textbf{x},\textbf{p}_{x})] d\textbf{x}d\textbf{p}_\textbf{x},
\label{Eq:PPF}
\end{equation}
Applying the chain rule:
\begin{equation}
\frac{\partial{Q(\lambda)}}{\partial{\lambda}} = -\frac{\beta}{{h}^{3N}N!} \iint \frac{\partial{H(\textbf{x},\textbf{p}_{x})}}{\partial{\lambda}}\exp[-\beta H(\textbf{x},\textbf{p}_{x})] d\textbf{x}d\textbf{p}_\textbf{x},
\label{Eq:PPF2}
\end{equation}
Substituting back into the expression for $\frac{\partial{A}}{\partial{\lambda}}$ gives:
\begin{align}
\frac{\partial{A(\lambda)}}{\partial{\lambda}} =& \frac{1}{{h}^{3N}N!}\frac{1}{Q(\lambda)} \iint \frac{\partial{H(\textbf{x},\textbf{p}_{x})}}{\partial{\lambda}}\exp[-\beta H(\textbf{x},\textbf{p}_{x})] d\textbf{x}d\textbf{p}_\textbf{x}, \\
=& \iint \frac{\partial{H(\textbf{x},\textbf{p}_{x})}}{\partial{\lambda}}\frac{\exp[-\beta H(\textbf{x},\textbf{p}_{x})]}{Q(\lambda)} d\textbf{x}d\textbf{p}_\textbf{x}, \\
=& \left \langle \frac{\partial{H(\textbf{x},\textbf{p}_{x}, \lambda)}}{\partial{\lambda}} \right \rangle_{\lambda}
\label{Eq:PA2}
\end{align}
Thus, the fundamental \textbf{TI} formula:
\begin{equation}
\Delta A = \int_{\lambda=0}^{\lambda=1}\left \langle \frac{\partial{H(\textbf{x},\textbf{p}_{x},\lambda)}}{\partial{\lambda}} \right \rangle_{\lambda} d\lambda
\label{Eq:TI}
\end{equation} 
where $\left \langle \cdots \right \rangle _{\lambda}$ corresponds to the ensemble average obtained using the Hamiltonian $H(\lambda)$. In practice, the ensemble of configurations can be obtained by molecule dynamic or Monte Carlo simulations. It is common practice in free energy calculations to use the coupling parameter $\lambda$ for defining the transformation from an initial $A$, with Hamiltonian $H_{A}$, to a final state $B$, with Hamiltonian $H_{B}$. The simplest coupling is linear scaling:
\begin{equation}
H(\lambda) = (1-\lambda) H_{A} + \lambda H_{B}
\end{equation}
Eq.~\ref{Eq:TI} represents the so-called thermodynamic integration formula. The exact calculation of $\Delta A$ would require an infinite number of ensemble average for $\lambda$ ranging from 0 to 1.
Therefore, the integral in Eq.~\ref{Eq:TI} needs to be approximated, e.g., by a summation over a discrete number of points $\lambda_{i}$,\cite{MordasiniJPCB2000} leading to 
\begin{equation}
\Delta A = \sum_{i}^{}\left \langle \frac{\partial{H(\lambda)}}{\partial{\lambda}} \right \rangle_{\lambda_{i}} \Delta\lambda_{i}
\label{Eq:dTI}
\end{equation} 
A finite number of $\lambda_{i}$ values between 0 and 1 is chosen and for each of them a complete molecule dynamic simulation is carried out resulting in an ensemble of configurations generated with $H(\lambda_{i})$.
The ensemble average of the derivative of the Hamiltonian with respect to $\lambda$ is then calculated for each $\lambda = \lambda_{i}$.
	
The accuracy of TI integral formula depends the exact method of numerical integration. In addition to summation method, the simplest numerical integration is to evaluate the integrand at the midpoint:
\begin{equation}
\Delta A \simeq \left \langle \frac{\partial{H(\lambda)}}{\partial{\lambda}} \right \rangle_{\lambda=\frac{1}{2}}
\label{Eq:TI1}
\end{equation} 
This might be a good first thing to do to get some picture of what is going on, but is only depends to be accurate for very smooth or small changes. %Gaussian quadrature formulas of higher order are generally more useful:
%\begin{equation}
%\Delta A = \sum_{i}^{} w_{i}\left \langle \frac{\partial{H(\lambda)}}{\partial{\lambda}} \right \rangle_{\lambda_{i}}
%\label{Eq:Gauss}
%\end{equation} 
%Some weights and quadrature points are given in the table~\ref{tab:Gauss}; other formulas are possible, \cite{HummerJCP1996} but the Gaussian one listed here are probably the most useful. The formulas are always symmetrical about $\lambda = 0.5$, so that $\lambda$ and $(1-\lambda)$ both have the same weight.
	
%\clearpage
%\begin{table*}
%	\caption{\label{tab:Gauss}Abscissas and weights for Gaussian integration.}
%	\newcommand{\rb}[1]{\raisebox{1.5ex}[0t]{#1}}
%	\begin{tabular}{lcccccccccccccccccc}
%		\hline
%	   n&$\lambda_{i}$&$1-\lambda_{i}$&$w_{i}$ \\
%		\hline
%	   1&     0.5&        & 1.0 \\
%		\hline
%	   2& 0.21132& 0.78867& 0.5 \\
%		\hline
%	   3&  0.1127& 0.88729& 0.27777 \\
%		&     0.5&        & 0.44444 \\
%		\hline
%	   5& 0.04691& 0.95308& 0.11846 \\
%		& 0.23076& 0.76923& 0.23931 \\
%		&     0.5&        & 0.28444 \\
%		\hline
%	   7& 0.02544& 0.97455& 0.06474 \\
%		& 0.12923& 0.87076& 0.13985 \\
%		& 0.29707& 0.70292& 0.19091 \\
%		&     0.5&        & 0.20897 \\
%		\hline
%	   9& 0.01592& 0.98408& 0.04064 \\
%		& 0.08198& 0.91802& 0.09032 \\
%		& 0.19331& 0.80669& 0.13031 \\
%		& 0.33787& 0.66213& 0.15617 \\
%		&     0.5&        & 0.16512 \\
%		\hline
%	  12& 0.00922& 0.99078& 0.02359 \\ 
%		& 0.04794& 0.95206& 0.05347 \\
%		& 0.11505& 0.88495& 0.08004 \\
%		& 0.20634& 0.79366& 0.10158 \\
%		& 0.31608& 0.68392& 0.11675 \\
%		& 0.43738& 0.56262& 0.12457 \\
%		\hline
%	\end{tabular} 
%\end{table*}