%%%%%%%%%%%%%%%%
% NEW CHAPTER! %
%%%%%%%%%%%%%%%%
\chapter{Introduction\label{chapter:introduction}}

\begin{chapquote}{Albert Einstein, %\textit{\url{https://en.wikiquote.org/wiki/Albert_Einstein}}
	}
``Everything should be made as simple as possible but not simpler.''
\end{chapquote}

Computer simulations of biological system have made much progress in the past decades. 

We are still facing many difficulties in three aspects, i.e. Hamiltonians, sampling efficiency and postprocessing methods.\cite{NielsJCTC2014}

In this booklet, we will not cover the whole spectrum of methods for enhanced samplings and free energy calculations, but only summarize some basic ideas. 
More complicated implementations of these methods, for instance 2-dimensional replica exchange molecular dynamics simulations, will not be discussed.

Recently, there is one special issue focusing on the methodologies of free energy calculations on Journal of Chemical Theory and Computation (Free Energy Calculations: Three Decades of Adventure in Chemistry and Biophysics, Journal of Chemical Theory and Computation, Volume 10, Issue 7, 2014, \url{http://pubs.acs.org/toc/jctcce/10/7}). 

There are also two books on this topic you might be interested in:
\begin{itemize}
\item Free Energy Calculations: Theory and Applications in Chemistry and Biology, Editors: Christophe Chipot, Andrew Pohorille, ISBN 978-3-540-38448-9, Springer-Verlag Berlin Heidelberg, 2007
\item Free Energy Computations: A Mathematical Perspective, Author: Tony Lelievre, Gabriel Stoltz, Mathias Rousset, ISBN-13: 978-1848162471, Imperial College Press, 2010
\end{itemize}

Before we get into the major content of this booklet, we would like to review some fundamental quantities in statistical mechanics. The first one is the partition function $Q$ for Hamiltonian $H(\mathbf{x},\mathbf{p}_{x})$, which is defined as
\begin{align}
  Q(N,V,T)=&\frac{1}{{h}^{3N}N!} \iint \exp[-\beta H(\mathbf{x},\mathbf{p}_{x})] d\mathbf{x}d\mathbf{p}_\mathbf{x}\notag\\
         =&\frac{1}{\Lambda^{3N}N!}Z(N,V,T),
\end{align}
where $\mathbf{x}$ and $\mathbf{p}_{x}$ are the coordinates and the conjugate momenta, respectively,
\begin{equation}
	Z(N,V,T)=\int \exp{\left(-\beta U(\mathbf{x})\right)}d\mathbf{x}
\end{equation}
is the configurational integral, $\Lambda$ is the temperature-dependent de Broglie wavelength, and $U(\mathbf{x})$ is the potential energy.

The partition function $Q$ can also be defined in energy space as
\begin{equation}
	Q(N,V,T)=\int \exp{\left(-\beta E\right)}\varOmega_{tot}(N,V,E)dE,
\end{equation}
where
\begin{equation}
	\varOmega_{tot}(N,V,E)=\frac{1}{{h}^{3N}N!}\iint_{V^N} \delta(H(\mathbf{x},\mathbf{p}_{x})-E)d\mathbf{x}d\mathbf{p}_\mathbf{x}
\end{equation}
is the complete density of states. Correspondingly, we can also define the configurational density of state as
\begin{equation}
	\varOmega_{con}\propto\frac{1}{N!}\int_{V^N} \delta(U(\mathbf{x})-E)d\mathbf{x}.
\end{equation}