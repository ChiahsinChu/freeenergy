\chapter{Enhanced Sampling\label{chapter:ES}}
\begin{chapquote}{Bernard R. Brooks%, \textit{\url{https://en.wikiquote.org/wiki/Albert_Einstein}}
	}
	``Keep the smart guys around you.''
\end{chapquote}
From the definition, the free energy of a specific system is dominated by phase space regions with low energy. However, these regions might be separated by high energy barriers. Transitions among these potential energy wells are often hindered by these barriers. According to the Boltzmann's Law, the probability of a sample $\mathbf{R}$ being visited is proportional to the Boltzmann's factor $\exp{(-\beta E(\mathbf{R}))}$, where $\beta=1/kT$ is called the inverse temperature. $k$ is the Boltzmann constant and $T$ is the temperature. According to some experience, in a 100 ns simulation, the system can overcome a barrier of 10 kT, which is 6 kcal/mol at room temperature (300 K). If the barrier is 3 kT higher, it takes the system about 1 $\mu$s (10 times longer) in average to go over the barrier. If the barrier height reaches 9 kcal/mol, it takes 10 $\mu$s at 300 K. And so on. With modern computers, the longest all-atom molecular dynamics simulation for biological systems is probably done by D.E. Shaw, which was on a time scale of 1 ms on a special-purpose computer ``Anton''. For most classical molecular dynamics simulations, the time scales are normally several $\mu$s to tens of $\mu$s. For simulations using expensive Hamiltonians, such as in QM/MM simulations, the time scales that can be reached are usually more than three orders shorter. Clearly, molecular dynamics simulations are plagued by a timescale problem. In order to observe abundant transitions among these energy minima, which is required by free energy calculations, enhanced samplings are often indispensable. As shown in the Boltzmann's factor, the essential quantity that determines the rate of transitions is $\beta E$. In order to accelerate the phase space sampling, we can either increase the temperature or decrease the energy barrier. All the methods shown below can be classified into these two categories. 
\clearpage 
\input chapters/REMD.tex
\clearpage 
\input chapters/US.tex
\clearpage 
\input chapters/lambdadynamics.tex
\clearpage 
\input chapters/metadynamics.tex
\clearpage 
\input chapters/OSRW.tex
\clearpage
\input chapters/EDS.tex